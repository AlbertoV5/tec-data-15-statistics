% Created 2022-10-13 Thu 22:47
% Intended LaTeX compiler: pdflatex
\documentclass[11pt]{article}
\usepackage[utf8]{inputenc}
\usepackage[T1]{fontenc}
\usepackage{graphicx}
\usepackage{longtable}
\usepackage{wrapfig}
\usepackage{rotating}
\usepackage[normalem]{ulem}
\usepackage{amsmath}
\usepackage{amssymb}
\usepackage{capt-of}
\usepackage{hyperref}
\author{Alberto Valdez}
\date{\today}
\title{Challenge\\\medskip
\large AutosRUS Analysis}
\hypersetup{
 pdfauthor={Alberto Valdez},
 pdftitle={Challenge},
 pdfkeywords={},
 pdfsubject={},
 pdfcreator={Emacs 28.1 (Org mode 9.6)}, 
 pdflang={English}}
\begin{document}

\maketitle
\tableofcontents


\section{Part 1: Predict MPG}
\label{sec:orgd397256}

This document is a complementary writeup done in the stye of ``literate programming'' for writing the R code and gathering the results.\\

For the report, go to \url{./readme.org} or README.md.\\

\begin{verbatim}
library(dplyr)
library(ggplot2)
library(tidyverse)
\end{verbatim}

\subsection{Import Data}
\label{sec:orge619fc2}

\begin{verbatim}
mpgcar <-
  read.csv(
    'MechaCar_mpg.csv',
    check.names = F,
    stringsAsFactors = F
)
head(mpgcar)
\end{verbatim}

\begin{org}
\begin{center}
\begin{tabular}{rrrrrr}
vehicle\_length & vehicle\_weight & spoiler\_angle & ground\_clearance & AWD & mpg\\
\hline
14.69709536 & 6407.94647 & 48.78998258 & 14.64098303 & 1 & 49.04918045\\
12.53420597 & 5182.080571 & 90 & 14.36667939 & 1 & 36.76606309\\
20 & 8337.981208 & 78.63232282 & 12.25371141 & 0 & 80\\
13.42848546 & 9419.670939 & 55.93903153 & 12.98935921 & 1 & 18.9414895\\
15.44997974 & 3772.666826 & 26.12816424 & 15.10396274 & 1 & 63.82456769\\
14.45356979 & 7286.594508 & 30.58567612 & 13.10695343 & 0 & 48.54267684\\
\end{tabular}
\end{center}
\end{org}

\subsection{Linear regression}
\label{sec:orgd80262c}

\begin{verbatim}
lm(
  mpg ~ vehicle_length +
  vehicle_weight +
  spoiler_angle +
  ground_clearance +
  AWD,
  data = mpgcar
)
\end{verbatim}

\begin{org}


Call:\\
lm(formula = mpg \textasciitilde{} vehicle\_length + vehicle\_weight + spoiler\_angle +\\
    ground\_clearance + AWD, data = mpgcar)\\

Coefficients:\\
     (Intercept)    vehicle\_length    vehicle\_weight     spoiler\_angle\\
      -1.040e+02         6.267e+00         1.245e-03         6.877e-02\\
ground\_clearance               AWD\\
       3.546e+00        -3.411e+00\\
\end{org}

\begin{verbatim}
summary(
  lm(
    mpg ~ vehicle_length +
    vehicle_weight +
    spoiler_angle +
    ground_clearance +
    AWD,
    data = mpgcar
  )
)
\end{verbatim}

\begin{org}


Call:\\
lm(formula = mpg \textasciitilde{} vehicle\_length + vehicle\_weight + spoiler\_angle +\\
    ground\_clearance + AWD, data = mpgcar)\\

Residuals:\\
     Min       1Q   Median       3Q      Max\\
-19.4701  -4.4994  -0.0692   5.4433  18.5849\\

Coefficients:\\
                   Estimate Std. Error t value Pr(>|t|)\\
(Intercept)      -1.040e+02  1.585e+01  -6.559 5.08e-08 \textbf{*}\\
vehicle\_length    6.267e+00  6.553e-01   9.563 2.60e-12 \textbf{*}\\
vehicle\_weight    1.245e-03  6.890e-04   1.807   0.0776 .\\
spoiler\_angle     6.877e-02  6.653e-02   1.034   0.3069\\
ground\_clearance  3.546e+00  5.412e-01   6.551 5.21e-08 \textbf{*}\\
AWD              -3.411e+00  2.535e+00  -1.346   0.1852\\
---\\
Signif. codes:  0 ‘***’ 0.001 ‘**’ 0.01 ‘*’ 0.05 ‘.’ 0.1 ‘ ’ 1\\

Residual standard error: 8.774 on 44 degrees of freedom\\
Multiple R-squared:  0.7149,	Adjusted R-squared:  0.6825\\
F-statistic: 22.07 on 5 and 44 DF,  p-value: 5.35e-11\\
\end{org}

\section{Part 2: Trip Analysis Visualization}
\label{sec:org9d7923a}

\begin{verbatim}
coildata <-
   read.csv(
    'Suspension_Coil.csv',
    check.names = F,
    stringsAsFactors = F
)
head(coildata)
\end{verbatim}

\begin{org}
\begin{center}
\begin{tabular}{llr}
VehicleID & Manufacturing\_Lot & PSI\\
\hline
V40858 & Lot1 & 1499\\
V40607 & Lot1 & 1500\\
V31443 & Lot1 & 1500\\
V6004 & Lot1 & 1500\\
V7000 & Lot1 & 1501\\
V17344 & Lot1 & 1501\\
\end{tabular}
\end{center}
\end{org}

Write an RScript that creates a total\_summary dataframe using the summarize() function to get the mean, median, variance, and standard deviation of the suspension coil’s PSI column.\\

\begin{verbatim}
total_summary <-
  coildata %>%
  summarize(
    Mean=mean(PSI),
    Median=median(PSI),
    Variance=var(PSI),
    SD=sd(PSI)
)
\end{verbatim}

\begin{org}
\begin{center}
\begin{tabular}{rrrr}
Mean & Median & Variance & SD\\
\hline
1498.78 & 1500 & 62.2935570469799 & 7.89262675203762\\
\end{tabular}
\end{center}
\end{org}

\begin{verbatim}
lot_summary <-
  coildata %>%
  group_by(Manufacturing_Lot) %>%
  summarize(
    Mean=mean(PSI),
    Median=median(PSI),
    Variance=var(PSI),
    SD=sd(PSI),
    .groups='keep'
  )
\end{verbatim}

\begin{org}
\begin{center}
\begin{tabular}{lrrrr}
Manufacturing\_Lot & Mean & Median & Variance & SD\\
\hline
Lot1 & 1500 & 1500 & 0.979591836734694 & 0.989743318610787\\
Lot2 & 1500.2 & 1500 & 7.46938775510204 & 2.73301806710128\\
Lot3 & 1496.14 & 1498.5 & 170.28612244898 & 13.0493724925369\\
\end{tabular}
\end{center}
\end{org}

\section{Part 3: T-Tests on Suspension Coils}
\label{sec:org1cfc598}

In your MechaCarChallenge.RScript, write an RScript using the t.test() function to determine if the PSI across all manufacturing lots is statistically different from the population mean of 1,500 pounds per square inch.\\

\begin{verbatim}
t.test(
  coildata$PSI,
  mu=mean(coildata$PSI)
)
\end{verbatim}

\begin{org}


One Sample t-test\\

data:  coildata\$PSI\\
t = 0, df = 149, p-value = 1\\
alternative hypothesis: true mean is not equal to 1498.78\\
95 percent confidence interval:\\
 1497.507 1500.053\\
sample estimates:\\
mean of x\\
  1498.78\\
\end{org}

Next, write three more RScripts in your MechaCarChallenge.RScript using the t.test() function and its subset() argument to determine if the PSI for each manufacturing lot is statistically different from the population mean of 1,500 pounds per square inch.\\

\begin{verbatim}
t.test(
  subset(
    coildata$PSI,
    coildata$Manufacturing_Lot == "Lot1"
  ),
  mu=mean(coildata$PSI)
)
\end{verbatim}

\begin{org}


One Sample t-test\\

data:  subset(coildata\$PSI, coildata\$Manufacturing\_Lot == ``Lot1'')\\
t = 8.7161, df = 49, p-value = 1.568e-11\\
alternative hypothesis: true mean is not equal to 1498.78\\
95 percent confidence interval:\\
 1499.719 1500.281\\
sample estimates:\\
mean of x\\
     1500\\
\end{org}

\begin{verbatim}
t.test(
  subset(
    coildata$PSI,
    coildata$Manufacturing_Lot == "Lot2"
  ),
  mu=mean(coildata$PSI)
)
\end{verbatim}

\begin{org}


One Sample t-test\\

data:  subset(coildata\$PSI, coildata\$Manufacturing\_Lot == ``Lot2'')\\
t = 3.6739, df = 49, p-value = 0.0005911\\
alternative hypothesis: true mean is not equal to 1498.78\\
95 percent confidence interval:\\
 1499.423 1500.977\\
sample estimates:\\
mean of x\\
   1500.2\\
\end{org}

\begin{verbatim}
t.test(
  subset(
    coildata$PSI,
    coildata$Manufacturing_Lot == "Lot3"
  ),
  mu=mean(coildata$PSI)
)
\end{verbatim}

\begin{org}


One Sample t-test\\

data:  subset(coildata\$PSI, coildata\$Manufacturing\_Lot == ``Lot3'')\\
t = -1.4305, df = 49, p-value = 0.1589\\
alternative hypothesis: true mean is not equal to 1498.78\\
95 percent confidence interval:\\
 1492.431 1499.849\\
sample estimates:\\
mean of x\\
  1496.14\\
\end{org}

\section{Complete script}
\label{sec:org30156b6}

Complete code for MechaCharChallenge.\\

\begin{verbatim}
library(dplyr)
library(ggplot2)
library(tidyverse)
mpgcar <-
  read.csv(
    'MechaCar_mpg.csv',
    check.names = F,
    stringsAsFactors = F
)
head(mpgcar)
lm(
  mpg ~ vehicle_length +
  vehicle_weight +
  spoiler_angle +
  ground_clearance +
  AWD,
  data = mpgcar
)
summary(
  lm(
    mpg ~ vehicle_length +
    vehicle_weight +
    spoiler_angle +
    ground_clearance +
    AWD,
    data = mpgcar
  )
)
coildata <-
   read.csv(
    'Suspension_Coil.csv',
    check.names = F,
    stringsAsFactors = F
)
head(coildata)
total_summary <-
  coildata %>%
  summarize(
    Mean=mean(PSI),
    Median=median(PSI),
    Variance=var(PSI),
    SD=sd(PSI)
)
lot_summary <-
  coildata %>%
  group_by(Manufacturing_Lot) %>%
  summarize(
    Mean=mean(PSI),
    Median=median(PSI),
    Variance=var(PSI),
    SD=sd(PSI),
    .groups='keep'
  )
t.test(
  coildata$PSI,
  mu=mean(coildata$PSI)
)
t.test(
  subset(
    coildata$PSI,
    coildata$Manufacturing_Lot == "Lot1"
  ),
  mu=mean(coildata$PSI)
)
t.test(
  subset(
    coildata$PSI,
    coildata$Manufacturing_Lot == "Lot2"
  ),
  mu=mean(coildata$PSI)
)
t.test(
  subset(
    coildata$PSI,
    coildata$Manufacturing_Lot == "Lot3"
  ),
  mu=mean(coildata$PSI)
)
\end{verbatim}

\begin{verbatim}
<<rscript>>
\end{verbatim}
\end{document}
